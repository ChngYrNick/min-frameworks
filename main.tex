\documentclass[a4paper,14pt]{extarticle} % 14й шрифт
%%% Преамбула %%%

\usepackage{fontspec} % XeTeX
\usepackage{xunicode} % Unicode для XeTeX

% Шрифты, xelatex
\defaultfontfeatures{Ligatures=TeX}
\setmainfont{Times New Roman} % Нормоконтроллеры хотят именно его
\newfontfamily\cyrillicfont{Times New Roman}
%\setsansfont{Liberation Sans} % Тут я его не использую, но если пригодится
\setmonofont{FreeMono} % Моноширинный шрифт для оформления кода

\usepackage{amssymb,amsfonts,amsmath} % Математика

\usepackage{enumerate} % Тонкая настройка списков
\usepackage{indentfirst} % Красная строка после заголовка
\usepackage{float} % Расширенное управление плавающими объектами
\usepackage{multirow} % Сложные таблицы
\usepackage{cite}
\usepackage{layouts}
\usepackage{tabularx}
\usepackage[table]{xcolor}

\colorlet{customGreen}{green!70!yellow!40}
\colorlet{tableGreen}{customGreen!80}
\colorlet{tableRed}{red!30!}

% Отступы у страниц
\usepackage{geometry}
\geometry{left=2.5cm}
\geometry{right=1.5cm}
\geometry{top=2cm}
\geometry{bottom=2cm}

% Русский язык
\usepackage{polyglossia}
\setdefaultlanguage{ukrainian}
\setotherlanguages{english,russian}
\PolyglossiaSetup{ukrainian}{indentfirst=true}

% Пути к каталогам с изображениями
\usepackage{graphicx} % Вставка картинок и дополнений
\graphicspath{{images/}{images/userguide/}{images/testing/}{images/infrastructure/}{extra/}{extra/drafts/}}

% Гиперссылки
\usepackage{hyperref}
\hypersetup{
    colorlinks, urlcolor={blue}, % Все ссылки черного цвета, кликабельные
    linkcolor={black}, citecolor={blue}, filecolor={black},
    pdfauthor={Serhii Moskovko},
    pdftitle={Modern minimal JavaScript frameworks}
}

\sloppy             % Избавляемся от переполнений
\hyphenpenalty=1000 % Частота переносов
\clubpenalty=10000  % Запрещаем разрыв страницы после первой строки абзаца
\widowpenalty=10000 % Запрещаем разрыв страницы после последней строки абзаца

% Списки
\usepackage{enumitem}

% Формат подрисуночных надписей
\addto\captionsukrainian{\renewcommand{\figurename}{Рисунок}} % Имя фигуры

% Пользовательские функции
\newcommand{\addimg}[4]{ % Добавление одного рисунка
    \begin{figure}
        \centering
        \includegraphics[width=#2\linewidth]{#1}
        \caption{#3} \label{#4}
    \end{figure}
}
\newcommand{\addimghere}[4]{ % Добавить рисунок непосредственно в это место
    \begin{figure}[H]
        \centering
        \includegraphics[width=#2\linewidth]{#1}
        \caption{#3} \label{#4}
    \end{figure}
}
\newcommand{\addtwoimghere}[5]{ % Вставка двух рисунков
    \begin{figure}[H]
        \centering
        \includegraphics[width=#2\linewidth]{#1}
        \hfill
        \includegraphics[width=#3\linewidth]{#2}
        \caption{#4} \label{#5}
    \end{figure}
}
\newcommand{\addimgapp}[3]{ 
    \begin{figure}[H]
        \centering
        \includegraphics[width=1\linewidth]{#1}
        \caption*{#2} \label{#3}
    \end{figure}
}

\usepackage[compact,explicit]{titlesec}

\newcommand{\anonsection}[1]{
    \phantomsection % Корректный переход по ссылкам в содержании
    \paragraph{\centerline{\MakeUppercase{#1}}\vspace{2\baselineskip}}
}

\pagestyle{empty}
 % Підключаємо преамбулу

%%% Початок документу
\begin{document}

\anonsection{Modern minimal JavaScript frameworks}

\begin{tabularx}{\textwidth}
  {|p{8em}|>{\columncolor{tableGreen}}X|>{\columncolor{tableRed}}X|}

\hline
\multicolumn{1}{|c|}{\bfseries{Framework}} & 
\multicolumn{1}{c|}{\bfseries{Pros}} &
\multicolumn{1}{c|}{\bfseries{Cons}} \\
\hline
\multirow{1}*{\href{https://github.com/lit/lit/}{Lit}} 
                   & Присутні шаблони для створення проекту з нуля 
                   & На данний момент SSR непридатний до корпоративних проектів \\
\cline{2-3}
                   & Невеликий за розміром кінцевий бандл (5 KB) &  \\
\cline{2-3}
                   & Додає можливість створювати власні html елементи з нестандартною поведінкою 
                   &  \\

\hline
\multirow{1}*{\href{https://github.com/bigskysoftware/htmx}{htmx}} 
                   & Є приклади з використанням фреймворку
                   & Важко будувати складні інтерфейси \\
\cline{2-3}
                   & Орієнтований на мінімізацію JavaScript коду 
                   &  \\

\hline
\multirow{1}*{\href{https://github.com/kasta-ua/twinspark-js}{TwinSpark.js}} 
                   & Невеликий за об'ємом коду (1400 стрічок) 
                   & Замало контрибьютерів. Підтримується комерційною компанією \\
\cline{2-3}
                   & Відсутні залежності від інших бібліотек 
                   & Важко будувати складні інтерфейси \\
\cline{2-3}
                   & Розроблений корпоративний \href{https://kasta.ua}{проект} на основі цього фреймворку 
                   &  \\
\cline{2-3}
                   & Орієнтований на мінімізацію JavaScript коду 
                   &  \\
\hline
\end{tabularx}

\end{document}
%%% Кінець документу
